\documentclass[../main/main]{subfiles}

\setcounter{chapter}{6}
%\setcounter{page}{1}
\begin{document}
\chapter{結論}
\section{まとめ}
\quad 本研究では,実数値遺伝的アルゴリズム(実数値GA)における設計変数の離散化が探索性能に与える影響を幅広く評価した.
また,設計変数の離散化が探索性能に与える影響を活用し,効率的探索が期待できる手法を提案した.
ここで,設計変数の離散化の影響評価に際しては,二つの代表的な実数値GA(NSGA-II,MOEA/D)と19個のベンチマーク問題を用いて,解の収束性と多様性の観点から評価を行った.
比較対象には小数点以下の桁数を2,4,6,8,16桁に制御したものを使用し,解の評価指標としては解の収束性を測る指標である Generational Distance (GD) と収束性と多様性を同時に測る総合指標 Inverted Generational Distance (IGD) を用いた.
各制御桁数で得られたGDとIGDを用いて,有意水準5\%でWilcoxonの順位和検定を行い,統計的な有意差を確認した.

数値実験より,どちらの実数値GAにおいても,多くの問題で設計変数の桁数が小さいほど(粗く離散化するほど)解の収束性が向上する傾向があることが確認された.
これは桁数が小さいほど効率的探索を妨げる Dominance-Resistant Solutions (DRSs) の発生が抑制されるためであると考えられる.
一方で,解の多様性に関しては,一部の問題で桁数を小さくしすぎると(粗く離散化しすぎると)悪化してしまう場合があった.
%目的関数の非線形性が強く,感度の高い問題では,桁数が小さすぎると万遍なく解を表現するために必要な粒度が失われてしまうため,多様性が失われてしまったと考えられる.
このことから,問題において多様性を維持するための必要最低限の粒度が存在することが推測される.
以上の結果から,最適化を行う前に各設計変数を適切に離散化することは難しいものの,問題に応じて各設計変数に適切な離散化を施すことで,解の多様性を維持しながらも収束性を向上させることができると考えられる.
%以上の結果から,問題に応じて各設計変数に適切な離散化を施すことで,解の多様性を維持しながらも収束性を向上させることができると考えられる.
%しかしながら,最適化を行う前に各設計変数を適切に離散化することは難しく,課題となっていた.

そこで,各設計変数を適応的に離散化することにより,多様性を維持しながらも収束性を向上させることが期待できる手法を提案した.
提案手法では,設計変数空間の分布状態を評価することで,各設計変数に対し適応的に離散化を行った.
ここでは,設計変数空間の分布状態を評価する指標として,標準偏差を用いる方法と推定された確率密度関数を用いる方法を提案した.
比較対象を標準偏差を用いた適応的離散化手法,推定された確率密度関数を用いた適応的離散化手法,離散化制御を用いない場合のものとし,探索性能を比較した.
提案手法をNSGA-IIに適用し,24個のベンチマーク問題を用いて,影響評価と同様の検証方法を用いて性能評価を行った.
%比較対象には標準偏差を

計算結果より,どちらの提案手法も離散化制御なしと比較し,ほとんどの問題で同等かそれ以上の収束性を示した.
また,多様性に関しては,どちらの提案手法も半数以上の問題で同等かそれ以上のIGD値が得られ,非劣解集合においても離散化制御なしと比較し大きな差は見られなかったことから,同程度の多様性が得られることが分かった.
以上の結果から,適応的離散化手法を用いることで,多くの問題で多様性を維持しながらも収束性を向上させる傾向があることが分かった.

特徴的な結果が得られた問題の分析結果から,標準偏差を用いた適応的離散化手法では,細かく離散化すべき設計変数とそうでない設計変数を区別しながら最適化を進めていることが分かった.
したがって,最適化で得られた離散化に用いた分割数の推移の情報を用いることで,細かい粒度が必要な設計変数とそうでない設計変数が見つけられるなど,最適化に使えるだけでなく,有用な設計情報が得られる可能性がある.
推定された確率密度関数を用いた適応的離散化手法では,設計変数空間に密な分布が複数存在するような問題に対して,分布のピークを検出し,その付近を細かく離散化し探索を進められることが分かった.
また,試行ごとのばらつきを減らし,安定的な結果が得られることが示唆された.

%比較対象として,小数点以下の桁数を2桁,
%提案手法においては,NSGA-IIと24個のベンチマーク問題を用いて,同様の観点から性能評価を行った.
%解の評価指標として

%実数値遺伝的アルゴリズムにおける設計変数空間の離散化を用いた適応的離散化手法を提案した.
%提案手法では,設計変数空間の分布状態を評価することで,各設計変数に対し適応的に離散化を行った.
%ここでは,設計変数空間の分布状態を評価する指標として,標準偏差を用いる方法と推定された確率密度関数を用いる方法を提案した.
%代表的な実数値遺伝的アルゴリズムであるNSGA-IIに提案手法を適用し,24個の多目的ベンチマーク問題を用いた.
%比較対象には標準偏差を用いた適応的離散化手法,推定された確率密度関数を用いた適応的離散化手法,離散化制御しなかった場合のものを用いた.
%解の評価指標として,解の収束性を測る指標であるGenerational Distance (GD) と収束性と多様性を同時に測る総合指標 Inverted Generational Distance (IGD) を用いた.
%各提案手法と,離散化制御なしで得られたGDとIGDを用いて,有意水準5\%でWilcoxonの順位和検定を行い,有効性を検証した.
%
%計算結果より,どちらの提案手法も離散化制御なしと比較し,ほとんどの問題で同等かそれ以上のGD値となり,高い収束性能を示した.
%また,IGDに関しては,どちらの提案手法も離散化制御なしと比較し,半数以上の問題において,同等化それ以上のIGD値となった.
%以上の結果から,適応的離散化手法を用いることで,多様性をある程度維持しながらも,収束性を向上させる傾向があることが分かった.
%また,制約条件付きの問題や最適化に精度が必要な問題,複数の最適値が存在するような問題に対しても,制御なしと同等の多様性を保ちながら高い収束性を示した.
%収束性を向上させながらも,多様性を維持・向上できるという点から,提案手法は多目的最適化問題で効果的であると考えられる.
%
%特徴的な結果が得られた問題の分析結果から,標準偏差を用いた適応的離散化手法では,細かく離散化すべき設計変数とそうでない設計変数を区別しながら最適化を進めていることが分かった.
%したがって,最適化で得られた分割数の推移の情報を用いることで,細かい粒度が必要な設計変数とそうでない設計変数が見つけられるなど,最適化に使えるだけでなく,設計情報が得られる可能性がある.
%推定された確率密度関数を用いた適応的離散化手法では,設計変数空間に密な分布が複数存在するような問題に対して,分布のピークを検出し,その付近を細かく離散化し探索を進められることが分かった.
%また,試行ごとのばらつきを減らし,安定的な結果を得られることが示唆された.


\section{今後の課題}
\quad 本研究では,設計変数空間の解の分布状態に応じて適応的に離散化を行う手法を提案し,疎な分布が得られた場合は設計変数を粗く離散化するように手法の設計を行った.
しかしながら適応的離散化手法の一つとして提案した推定された確率密度関数を用いる手法では,疎な分布が得られた場合でも,部分的に細かく離散化を行う領域があるため,意図通り離散化の制御ができていない場合がある.
そのため,推定された確率密度関数を用いる手法で使用した離散化制御式を改善する必要があると考える.

\end{document}